
%%%%%%%%%%%%%%%%%%%%  Ссылки %%%%%%%%%%%%%%%%%%%%%%%%%%%%%%%%%%%%%%%%%
% 1) Нарисовать нужный символ и узнать код:
%     http://detexify.kirelabs.org/classify.html
% 2) Список символов:
%     http://tug.ctan.org/info/symbols/comprehensive/symbols-a4.pdf

%%%%%%%%%%%%%%%%%%%%% Важные моменты %%%%%%%%%%%%%%%%%%%%%%%%%%%%%%%%%
% 0) Все латинские (и греческие) буквы считаются
%     математическими символами и потому пишутся в
%     «математической моде», то есть между $...$ для
%     внутристрочных формул и \[...\] для выключных (не $$...$$, как
%     часто пишут).
% 1) Не забывайте про знаки препинания в формулах (точки, запятые).
% 2) Правильно: L^+, a^*, f\colon A \to B,  неправильно: L+, a*, f: A -> B.
% 3) Обращайте внимание на спелл-чекер (красное подчёркивание слов
%     с ошибками и неизвестных слов).
% 4) Пустое слово обозначается символом $\eps$, а не $e$.
% 5) В записи множеств наподобие $\{ h(w) \mid w \in L \}$
%     следует использовать команду \mid, а не |.
%     Обратите также внимание, что для печати символов { и }
%     необходимы ‘\’, иначе скобки понимаются как средство
%     группировки, например: a_{n-1} (n-1 это нижний индекс).

%%%%%%%%%%%%%%%%%%%%% Часто используемые символы %%%%%%%%%%%%%%%%
% Большинство значков используется в математическом режиме:
%     внутри $...$ или \[...\].
% - символ  |- это  \vdash.
% - греческие буквы: \alpha, \beta, \gamma, \eps,...
%     или заглавные: \Alpha, \Sigma, \Omega,...
% - принадлежит (элемент множеству): \in
% - индексы: a^n, L_i, L^+, L^*
% - объединение/пересечение:
%     двух множеств: A \cup B, A \cap B,
%     семейства множеств: \bigcup_{n \ge 0} L^n
% - нестрогие неравенства: \le, \ge,
% - пустое множество: \es
% - кавычки в тексте: <<текст>>
% - троеточие: \ldots, а не ...! Заполнение точками: \dotfill
% - угловые скобки: \langle \rangle
% - Для тире следует использовать --- а не один -
% - Для длины слова: | а не \mid.
% - Текст внутри формулы: \text.
% - Кавычки: не ", а <<ёлочки>> для русских слов и дважды ' ' для английских.
%   образец кавычек-ёлочек для русских слов «праволинейный»
% - Не пишите слова типа begin и liFe внутри $$: у них будут неверные пробелы между буквами,
%    используйте обычный курсив: \emph{здесь курсив}.
% - Для горизонтального отступа в формуле вместо mspace: \quad или \qquad (поменьше и побольше).
% - Следите за длинными формулами. Если формулы две, например, h(0)=a, h(1)=b, то нужно
%    ставить их в две разные пары долларов: $h(0)=a$, $h(1)=b$, а не в одну. Если формулы
%    длинная и плохо помещается на строке (выходит на поля), то делайте её выключной (т.е. \[...\]).
% - Следите за знаками препинания. В и~т.~д. желательно ставить вот такие ~ (неразрывный пробел).
% - разбивайте строки с помощью Enter на более короткие, иначе под Git'ом неудобно diff'ы смотреть.
%%%%%%%%%%%%%%%%%%%%% Крупные элементы %%%%%%%%%%%%%%%%%%%%%%%%%%%%
% Системы уравнений:
%\begin{equation}% нумерованное уравнение
%    \begin{cases}
%        X_1 = 1X_1 + 0X_2 + \es X_3; \\
%        X_2 = 1X_1 + \es X_2 + 0X_3; \\
%        X_3 = 0X_1 + \es X_2 + 1X_3.
%    \end{cases}
%\end{equation}
% Если нужна текстовая вставка внутри любой формулы
% (например, системы уравнений), она делается с помощью
% команды \text{текст}.
%
% Продукции:
%\begin{equation}
%\begin{array}{l}
%    S \to 1A \mid 2S; \\
%    A \to 0B \mid 0S \mid 1A; \\
%    B \to 1C \mid 2C; \\
%    C \to \eps \mid 1S \mid 2A.
%\end{array}
%\end{equation}
%
% Матрицы (с любыми скобками) и другие элементы:
% http://en.wikibooks.org/wiki/LaTeX/Mathematics#Matrices_and_arrays
%
% Нумерованные списки:
%\begin{enumerate}
%    \item пункт
%    \item и ещё…
%\end{enumerate}
