% !TEX TS-program = pdflatex
% !TEX encoding = UTF-8 Unicode
\documentclass[fleqn,12pt, a4paper]{article}
% Русский язык
\usepackage[T2A, T1]{fontenc}
\usepackage[utf8]{inputenc}
\usepackage[english,russian]{babel}

 % Поля страницы
\usepackage{geometry}
\geometry{left=2cm}
\geometry{right=1.5cm}
\geometry{top=2cm}
\geometry{bottom=2cm}

\usepackage[pdftex]{graphicx}
\graphicspath{{images/}}%путь к рисункам
\DeclareGraphicsExtensions{.png}

\usepackage{pgf}
\usepackage{tikz}
\usetikzlibrary{arrows,automata,positioning}
\usetikzlibrary{shapes.misc}

%\usepackage{indentfirst} % включить отступ у первого абзаца

\usepackage{booktabs} % for much better looking tables
\usepackage{array} % for better arrays (eg matrices) in maths
\usepackage{multirow}
\usepackage{paralist} % very flexible & customisable lists (eg. enumerate/itemize, etc.)
\usepackage{enumitem }
%\usepackage{amsmath}
\usepackage[fleqn]{amsmath}
\setlength{\mathindent}{0pt}
\usepackage{amsfonts,amssymb,amsthm,epsfig,epstopdf,titling,url,array,nccmath }

\usepackage{listings} % for source code highlighning
\lstset{language=C++}

% Переопределения
\newcommand {\la} {\langle}
\newcommand {\ra} {\rangle}
\renewcommand {\le} {\leqslant}
\renewcommand {\ge} {\geqslant}
\renewcommand {\leq} {\leqslant}
\renewcommand {\geq} {\geqslant}
\renewcommand {\emptyset} {\varnothing}
\newcommand {\es} {\varnothing}
\newcommand {\eps} {\varepsilon}
\newcommand {\mydef}[1]{\emph{#1}}
\newcommand{\To}{\Rightarrow}

\newcommand{\Sig}{\ensuremath{\Sigma}}
\newcommand{\N}{\ensuremath{\mathbb N}}
\newcommand{\NO}{\ensuremath{\mathbb N_0}}

% Неразрывный дефис, который допускает перенос внутри слов,
% типа жёлто-синий: нужно писать жёлто"/синий.
\makeatletter
    \defineshorthand[russian]{"/}{\mbox{-}\bbl@allowhyphens}
\makeatother
% Конец переопределений

\begin{document}

%%%%%%%%%%%%%%%%%%%%%%%%%%%%%%%%%%%%%%%
%						Титульный лист						%
%%%%%%%%%%%%%%%%%%%%%%%%%%%%%%%%%%%%%%%

\begin{titlepage}

\begin{center} 
\MakeUppercase{Минобрнауки России}\\
Федеральное государственное автономное образовательное учреждение
высшего профессионального образования
<<Южный федеральный университет>>\\[1 cm]

Институт математики, механики и компьютерных наук им. И.И. Воровича\\[1 cm]

Кафедра алгебры и дискретной математики\\[5cm]

\Huge \MakeUppercase{Курсовая работа} \\[0.6cm]

\large по предмету: <<Теория автоматов и формальных языков>>\\[7cm]

\end{center}

\begin{flushleft}
{\large 
Выполнил: \\
Студент 3 курса, $N$ группы \hfill И.О. Фамилия \\[1cm]

Проверил:\\
к.т.н., ст. преп. кафедры АДМ \hfill Е.В. Алымова}

\end{flushleft}
\vfill

\begin{center}
Ростов-на-Дону\\
2015
\end{center}
\end{titlepage}

%%%%%%%%%%%%%%%%%%%%  Ссылки %%%%%%%%%%%%%%%%%%%%%%%%%%%%%%%%%%%%%%%%%
% 1) Нарисовать нужный символ и узнать код:
%     http://detexify.kirelabs.org/classify.html
% 2) Список символов:
%     http://tug.ctan.org/info/symbols/comprehensive/symbols-a4.pdf

%%%%%%%%%%%%%%%%%%%%% Важные моменты %%%%%%%%%%%%%%%%%%%%%%%%%%%%%%%%%
% 0) Все латинские (и греческие) буквы считаются
%     математическими символами и потому пишутся в
%     «математической моде», то есть между $...$ для
%     внутристрочных формул и \[...\] для выключных (не $$...$$, как
%     часто пишут).
% 1) Не забывайте про знаки препинания в формулах (точки, запятые).
% 2) Правильно: L^+, a^*, f\colon A \to B,  неправильно: L+, a*, f: A -> B.
% 3) Обращайте внимание на спелл-чекер (красное подчёркивание слов
%     с ошибками и неизвестных слов).
% 4) Пустое слово обозначается символом $\eps$, а не $e$.
% 5) В записи множеств наподобие $\{ h(w) \mid w \in L \}$
%     следует использовать команду \mid, а не |.
%     Обратите также внимание, что для печати символов { и }
%     необходимы ‘\’, иначе скобки понимаются как средство
%     группировки, например: a_{n-1} (n-1 это нижний индекс).

%%%%%%%%%%%%%%%%%%%%% Часто используемые символы %%%%%%%%%%%%%%%%
% Большинство значков используется в математическом режиме:
%     внутри $...$ или \[...\].
% - символ  |- это  \vdash.
% - греческие буквы: \alpha, \beta, \gamma, \eps,...
%     или заглавные: \Alpha, \Sigma, \Omega,...
% - принадлежит (элемент множеству): \in
% - индексы: a^n, L_i, L^+, L^*
% - объединение/пересечение:
%     двух множеств: A \cup B, A \cap B,
%     семейства множеств: \bigcup_{n \ge 0} L^n
% - нестрогие неравенства: \le, \ge,
% - пустое множество: \es
% - кавычки в тексте: <<текст>>
% - троеточие: \ldots, а не ...! Заполнение точками: \dotfill
% - угловые скобки: \langle \rangle
% - Для тире следует использовать --- а не один -
% - Для длины слова: | а не \mid.
% - Текст внутри формулы: \text.
% - Кавычки: не ", а <<ёлочки>> для русских слов и дважды ' ' для английских.
%   образец кавычек-ёлочек для русских слов «праволинейный»
% - Не пишите слова типа begin и liFe внутри $$: у них будут неверные пробелы между буквами,
%    используйте обычный курсив: \emph{здесь курсив}.
% - Для горизонтального отступа в формуле вместо mspace: \quad или \qquad (поменьше и побольше).
% - Следите за длинными формулами. Если формулы две, например, h(0)=a, h(1)=b, то нужно
%    ставить их в две разные пары долларов: $h(0)=a$, $h(1)=b$, а не в одну. Если формулы
%    длинная и плохо помещается на строке (выходит на поля), то делайте её выключной (т.е. \[...\]).
% - Следите за знаками препинания. В и~т.~д. желательно ставить вот такие ~ (неразрывный пробел).
% - разбивайте строки с помощью Enter на более короткие, иначе под Git'ом неудобно diff'ы смотреть.
%%%%%%%%%%%%%%%%%%%%% Крупные элементы %%%%%%%%%%%%%%%%%%%%%%%%%%%%
% Системы уравнений:
%\begin{equation}% нумерованное уравнение
%    \begin{cases}
%        X_1 = 1X_1 + 0X_2 + \es X_3; \\
%        X_2 = 1X_1 + \es X_2 + 0X_3; \\
%        X_3 = 0X_1 + \es X_2 + 1X_3.
%    \end{cases}
%\end{equation}
% Если нужна текстовая вставка внутри любой формулы
% (например, системы уравнений), она делается с помощью
% команды \text{текст}.
%
% Продукции:
%\begin{equation}
%\begin{array}{l}
%    S \to 1A \mid 2S; \\
%    A \to 0B \mid 0S \mid 1A; \\
%    B \to 1C \mid 2C; \\
%    C \to \eps \mid 1S \mid 2A.
%\end{array}
%\end{equation}
%
% Матрицы (с любыми скобками) и другие элементы:
% http://en.wikibooks.org/wiki/LaTeX/Mathematics#Matrices_and_arrays
%
% Нумерованные списки:
%\begin{enumerate}
%    \item пункт
%    \item и ещё…
%\end{enumerate}

%%%%%%%%%%%%%%%%%%%%%%%%%%%%%%%%%%%%%%%
%						Текст курсовой						%
%%%%%%%%%%%%%%%%%%%%%%%%%%%%%%%%%%%%%%%

\section*{Задание 1}
Условие задания.
\begin{enumerate}[label=(\roman{*})]
	\item Построим ПЛ-грамматику...
	\item \ldots
	\item Пример задания автомата диаграммой переходов:\\
\begin{tikzpicture}[auto,>=stealth', node distance=3cm,auto,every state/.style={thick}]
	\node (init) {};
	\node[state] (q0) [right=.7cm of init] {$q_0$};
	\node[state] (q1) [above right of=q0] {$q_1$};
    \node[state] (q2) [right of=q0] {$q_2$};
    \node[state] (q3) [below right of=q0] {$q_3$};
	\node[state,accepting] (qf) [right of=q2, above right of=q3] {$q_f$};

	\path[->]
    	(init) edge (q0)
		(q0) edge [loop above] node {$1, 2, 3$} (q0)
		(q0) edge node {$1$} (q1)
        (q0) edge node {$2$} (q2)
        (q0) edge node {$3$} (q3)
        (q1) edge [loop above] node[right] {$1, 2, 3$} (q1)
        (q1) edge node {$1$} (qf)
        (q2) edge [loop above] node[right] {$1, 2, 3$} (q2)
        (q2) edge node {$2$} (qf)
        (q3) edge [loop above] node[right] {$1, 2, 3$} (q3)
        (q3) edge node {$3$} (qf);
\end{tikzpicture}
	\item Пример задания функции переходов таблицей:\\
		\begin{tabular}{llll}
		\toprule
		\multicolumn{1}{c}{\multirow{2}{*}{\Large $\delta^\prime$}}
			& \multicolumn{3}{c}{Вход} \\
		\cmidrule(rl){2-4}
			& \multicolumn{1}{c}{1}
				& \multicolumn{1}{c}{2}
				& \multicolumn{1}{c}{3} \\
		\midrule
		$\{0\}$         & $\{0;1\}$       & $\{0;2\}$       & $\{0;3\}$       \\
		$\{0;1\}$       & $\{0;1;f\}$     & $\{0;1;2\}$     & $\{0;1;3\}$     \\
		$\{0;2\}$       & $\{0;1;2\}$     & $\{0;2;f\}$     & $\{0;2;3\}$     \\
		$\{0;3\}$       & $\{0;1;3\}$     & $\{0;2;3\}$     & $\{0;3;f\}$     \\
		$\{0;1;2\}$     & $\{0;1;2;f\}$   & $\{0;1;2;f\}$   & $\{0;1;2;3\}$   \\
		$\{0;1;3\}$     & $\{0;1;3;f\}$   & $\{0;1;2;3\}$   & $\{0;1;3;f\}$   \\
		$\{0;1;f\}$     & $\{0;1;f\}$     & $\{0;1;2\}$     & $\{0;1;3\}$     \\
		$\{0;2;3\}$     & $\{0;1;2;3\}$   & $\{0;2;3;f\}$   & $\{0;2;3;f\}$   \\
		$\{0;2;f\}$     & $\{0;1;2\}$     & $\{0;2;f\}$     & $\{0;2;3\}$     \\
		$\{0;3;f\}$     & $\{0;1;3\}$     & $\{0;2;3\}$     & $\{0;3;f\}$     \\
		$\{0;1;2;f\}$   & $\{0;1;2;f\}$   & $\{0;1;2;f\}$   & $\{0;1;2;3\}$   \\
		$\{0;1;3;f\}$   & $\{0;1;3;f\}$   & $\{0;1;2;3\}$   & $\{0;1;3;f\}$   \\
		$\{0;2;3;f\}$   & $\{0;1;2;3\}$   & $\{0;2;3;f\}$   & $\{0;2;3;f\}$   \\
		$\{0;1;2;3\}$   & $\{0;1;2;3;f\}$ & $\{0;1;2;3;f\}$ & $\{0;1;2;3;f\}$ \\
		$\{0;1;2;3;f\}$ & $\{0;1;2;3;f\}$ & $\{0;1;2;3;f\}$ & $\{0;1;2;3;f\}$ \\ \bottomrule
		\end{tabular}
\end{enumerate}

\newpage
\section*{Задание 2}
Условие задания.
\begin{enumerate}[label=(\roman{*})]
	\item \ldots
\end{enumerate}

\newpage
\section*{Задание 3}
Условие задания.
\begin{enumerate}[label=(\roman{*})]
	\item \ldots
\end{enumerate}

\end{document}