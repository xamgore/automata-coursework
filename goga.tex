% !TEX TS-program = pdflatex
% !TEX encoding = UTF-8 Unicode
\documentclass[fleqn,12pt, a4paper]{article}
% Русский язык
\usepackage[T2A, T1]{fontenc}
\usepackage[utf8]{inputenc}
\usepackage[english,russian]{babel}

 % Поля страницы
\usepackage{geometry}
\geometry{left=2cm}
\geometry{right=1.5cm}
\geometry{top=2cm}
\geometry{bottom=2cm}

\usepackage[pdftex]{graphicx}
\graphicspath{{images/}}%путь к рисункам
\DeclareGraphicsExtensions{.png}

\usepackage{pgf}
\usepackage{tikz}
\usetikzlibrary{arrows,automata,positioning}
\usetikzlibrary{shapes.misc}

%\usepackage{indentfirst} % включить отступ у первого абзаца

\usepackage{booktabs} % for much better looking tables
\usepackage{array} % for better arrays (eg matrices) in maths
\usepackage{multirow}
\usepackage{paralist} % very flexible & customisable lists (eg. enumerate/itemize, etc.)
\usepackage{enumitem }
%\usepackage{amsmath}
\usepackage[fleqn]{amsmath}
\setlength{\mathindent}{0pt}
\usepackage{amsfonts,amssymb,amsthm,epsfig,epstopdf,titling,url,array,nccmath }

\usepackage{listings} % for source code highlighning
\lstset{language=C++}

% Переопределения
\newcommand {\la} {\langle}
\newcommand {\ra} {\rangle}
\renewcommand {\le} {\leqslant}
\renewcommand {\ge} {\geqslant}
\renewcommand {\leq} {\leqslant}
\renewcommand {\geq} {\geqslant}
\renewcommand {\emptyset} {\varnothing}
\newcommand {\es} {\varnothing}
\newcommand {\eps} {\varepsilon}
\newcommand {\mydef}[1]{\emph{#1}}
\newcommand{\To}{\Rightarrow}

\newcommand{\Sig}{\ensuremath{\Sigma}}
\newcommand{\N}{\ensuremath{\mathbb N}}
\newcommand{\NO}{\ensuremath{\mathbb N_0}}

% Неразрывный дефис, который допускает перенос внутри слов,
% типа жёлто-синий: нужно писать жёлто"/синий.
\makeatletter
    \defineshorthand[russian]{"/}{\mbox{-}\bbl@allowhyphens}
\makeatother
% Конец переопределений

\begin{document}

%%%%%%%%%%%%%%%%%%%%%%%%%%%%%%%%%%%%%%%
%						Титульный лист						%
%%%%%%%%%%%%%%%%%%%%%%%%%%%%%%%%%%%%%%%

\begin{titlepage}

\begin{center} 
\MakeUppercase{Минобрнауки России}\\
Федеральное государственное автономное образовательное учреждение
высшего профессионального образования
<<Южный федеральный университет>>\\[1 cm]

Институт математики, механики и компьютерных наук им. И.И. Воровича\\[1 cm]

Кафедра алгебры и дискретной математики\\[5cm]

\Huge \MakeUppercase{Курсовая работа} \\[0.6cm]

\large по предмету: <<Теория автоматов и формальных языков>>\\[7cm]

\end{center}

\begin{flushleft}
{\large 
Выполнил: \\
Студент 3 курса, 9 группы \hfill Стребежев Игорь \\[1cm]

Проверил:\\
к.т.н., ст. преп. кафедры АДМ \hfill Е.В. Алымова}

\end{flushleft}
\vfill

\begin{center}
Ростов-на-Дону\\
\the\year
\end{center}
\end{titlepage}

%%%%%%%%%%%%%%%%%%%%%%%%%%%%%%%%%%%%%%%
%		Текст курсовой				
%%%%%%%%%%%%%%%%%%%%%%%%%%%%%%%%%%%%%%%

\section*{Задание 1}

Вариант 18. Язык над алфавитом $\Sigma = \{ 0, 1 \} $, состоящий из всех слов, в которых после третьей слева единицы стоит четное число групп вида 01.

\begin{enumerate}[label=(\roman{*})]
	\item Построить ПЛ-грамматику $G$, порождающую $L$.
	
	\begin{equation}
		\begin{array}{l}
			S \to 0S \mid 1A, \\
			A \to 0A \mid 1B, \\
			B \to 0B \mid 1C \mid 1, \\
			C \to 0101C \mid 0101;
		\end{array}
	\end{equation}
	
	\item Доказать вложения $L \subseteq L(G)$, $L(G) \subseteq L$.\\
	
	\underline{$L \subseteq L(G)$}: Рассмотрим дерево выводов грамматики.
	
	$S \implies 0S \implies 00S \implies 000S \implies \dots$ --- нули разрешены в начале слов в $L$.
	
	$S \implies 1A \implies 10A \implies 100A \implies \dots$ --- генерация первой единицы и последовательности нулей.
	
	Первые три правила грамматики одинаковы и образуют три пары из последовательности (возможно пустой) нулей и единицы. Что также удовлетворяет описанию языка~$L$.
	
	$11B \implies 111 $ --- в таком случае количество групп вида 01 будет равно нулю (чётно), принадлежит языку.
	
	$11B \implies 111C \implies 111\ 0101$ --- генерация пары групп.
	
	$11B \implies 111C \implies 111\ 0101C \implies 111\ 0101\ 0101$ --- всегда чётное количество групп.
	
	Таким образом все слова, порождаемые грамматикой $G$ принадлежат языку $L$, то есть $L \subseteq L(G)$.\\
	
	
	\underline{$L(G) \subseteq L$}: Язык L имеет структуру $(0*1)(0*1)(0*1)(1010)*$, каждую цепочку этого языка можно получить применяя правила грамматики, как показано в предыдущем пункте, что и доказывает вложение.
	
	\item Путём решения системы линейных уравнений с регулярными коэффициентами построить регулярное выражение, описывающее $L$.
	
	
	\begin{equation}
		\begin{array}{l}
		S = 0S + 1A, \\
		A = 0A + 1B, \\
		B = 0B + 1C + 1, \\
		C = 0101C + 0101; \\\\
		
		C = (0101)^*0101 = 0101^+, \\
		B = 0B + 1(0101^+ + \eps) = 0^* 1(0101^+ + \eps), \\
		A = 0A + 1(0^* 1(0101^+ + \eps)) = 0^* 1(0^* 1(0101^+ + \eps)), \\
		S = 0S + 1(0^* 1(0^* 1(0101^+ + \eps))) = 0^* 1(0^* 1(0^* 1(0101^+ + \eps))); \\\\
		
		0^* 1(0^* 1(0^* 1(0101^+ + \eps))) = (0^*1) (0^*1) (0^*1) (0101)^*.
		\end{array}
	\end{equation}
	
	
	\item Построить НКА или \emph{e}-НКА $M^{ND}$, распознающий язык $L$, предъявить его граф. \\
	
	\begin{tikzpicture}[auto,>=stealth', node distance=3cm,auto,every state/.style={thick}]
		\node (init) {};
		\node[state] (qS) [right=.7cm of init] {$S$};
		\node[state] (qA) [right of=qS] {$A$};
		\node[state] (qB) [right of=qA] {$B$};
		\node[state] (qC) [right of=qB] {$C$};
		\node[state,accepting] (qF) [below of=qB] {$F$};
		\node[state,accepting] (qE) [below of=qC] {$E$};
		
		\path[->]
		(init) edge (qS)
		(qS) edge [loop above] node {$0$} (qS)
		(qS) edge node {$1$} (qA)
		(qA) edge [loop above] node {$0$} (qA)
		(qA) edge node {$1$} (qB)
		(qB) edge [loop above] node {$0$} (qB)
		(qB) edge node {$1$} (qF)
		(qB) edge node {$1$} (qC)
		(qC) edge [loop above] node {$0101$} (qC)
		(qC) edge node {$0101$} (qE)
		;
	\end{tikzpicture}
	
	\item Построить ДКА $M^D$ путём детерминизации $M^{ND}$, предъявить его граф. \\
	
	Построим таблицу переходов $M^{ND}$.
	
	\begin{tabular}{lcccccccccccc}
		\toprule
		\multicolumn{1}{c}{
			$\delta^\prime \mid$
		} 
		& \multicolumn{1}{c}{ $S$ }
		& \multicolumn{1}{c}{ $A$ }
		& \multicolumn{1}{c}{ $B$ }
		& \multicolumn{1}{c}{ $C$ }
		& \multicolumn{1}{c}{ $K^0$ }
		& \multicolumn{1}{c}{ $K^{01}$ }
		& \multicolumn{1}{c}{ $K^{010}$ }
		& \multicolumn{1}{c}{ $E^0$ }
		& \multicolumn{1}{c}{ $E^{01}$ }
		& \multicolumn{1}{c}{ $E^{010}$ }
		& \multicolumn{1}{c}{ $E$ }
		& \multicolumn{1}{c}{ $F$ }
		\\
		\midrule
		
		0 $\mid$ &
		$S$ & $A$ & $B$ & $\{K^0, E^0\}$ & --- & $K^{010}$ & --- & --- & $E^{010}$ & --- & --- & --- \\
		1 $\mid$ &
		$A$ & $B$ & $\{ C, F \}$ & --- & $K^{01}$ & --- & $C$ & $E^{01}$ & --- & $E$ & --- & ---	\\ 		
		\bottomrule
	\end{tabular}
	
	\vspace{1em}
	
	Расширим таблицу при приведении автомата в $M^D$.
	
	
	\begin{tabular}{lcccccccc}
		\toprule
		\multicolumn{1}{c}{
			$\delta^\prime \mid$
		} 
		& \multicolumn{1}{c}{ $S$ }
		& \multicolumn{1}{c}{ $A$ }
		& \multicolumn{1}{c}{ $B$ }
		& \multicolumn{1}{c}{ $\{C, F\}$ }
		& \multicolumn{1}{c}{ $\{K^{ 0 }, E^{ 0 }\}$ }
		& \multicolumn{1}{c}{ $\{K^{ 01 }, E^{ 01 }\}$ }
		& \multicolumn{1}{c}{ $\{K^{ 010 }, E^{ 010 }\}$ }
		& \multicolumn{1}{c}{ $\{ C, E \}$ }
		\\
		\midrule
		
		0 $\mid$ &
		$S$ & $A$ & $B$ &  $\{K^0, E^0\}$ & --- & $\{K^{ 010 }, E^{ 010 }\}$ & --- &  $\{K^{ 0 }, E^{ 0 }\}$  \\
		1 $\mid$ &
		$A$ & $B$ & $\{ C, F \}$ & --- &  $\{K^{ 01 }, E^{ 01 }\}$ & --- & $\{ C, E \}$ & ---	\\ 		
		\bottomrule
	\end{tabular}
	
	\vspace{1em} Отобразим этот автомат на диаграмме.
	
	
	\begin{tikzpicture}[auto,>=stealth', node distance=3cm,auto,every state/.style={thick}]
		\node (init) {};
		\node[state] (qS) [right=.7cm of init] {$S$};
		\node[state] (qA) [right of=qS] {$A$};
		\node[state] (qB) [right of=qA] {$B$};
		\node[state,accepting] (qCF) [right of=qB] {$CF$};
		\node[state] (qKE) [right of=qCF] {$K^0E^0$};
		\node[state,accepting] (qCE) [right of=qKE] {$CE$};
		
		\path[->]
		(init) edge (qS)
		(qS) edge [loop above] node {$0$} (qS)
		(qS) edge node {$1$} (qA)
		(qA) edge [loop above] node {$0$} (qA)
		(qA) edge node {$1$} (qB)
		(qB) edge [loop above] node {$0$} (qB)
		(qB) edge node {$1$} (qCF)
		(qCF) edge node {$0$} (qKE)
		
		(qKE) edge  [bend left] node {$101$} (qCE)
		(qCE) edge  [bend left] node {$0$} (qKE)
		;
	\end{tikzpicture}
\end{enumerate}

\newpage
\section*{Задание 2}
Вариант 29. $\Sigma = \{ 0, 1, 2, 3, 4, 5 \}$, $A = \{ 53150, 53555, 5510, 0001 \}$.

\begin{enumerate}
	\item Для каждого слова $w_i \in A$ построить НКА $M_i^{ND}$, распознающий наличие в произвольной строке $s \in \Sigma^*$ подстроки $w_i$.
	\item Для каждого НКА $M_i^{ND}$ построить соответствующий ДКА $M_i^{D}$.\\
	
	Слово 53150.\\
	
	\begin{tikzpicture}[auto,>=stealth', node distance=3cm,auto,every state/.style={thick}]
		\node (init) {};
		\node[state] (S) [right=.7cm of init] {$S$};
		\node[state] (qA1) [below of=qS] {$A^1$};
		\node[state] (qA2) [right of=qA1] {$A^2$};
		\node[state] (qA3) [right of=qA2] {$A^3$};
		\node[state] (qA4) [right of=qA3] {$A^4$};
		\node[state] (qA5) [right of=qA4] {$A^5$};
		\node[state,accepting] (qA6) [right of=qA5] {$A^6$};
		
		\path[->]
		(init) edge (qS)
		(qS) edge [loop above] node {$ 0, 1, 2, 3, 4, 5 $} (qS)
		(qS) edge node {$\eps$} (qA1)
		(qA1) edge node {$5$} (qA2)
		(qA2) edge node {$3$} (qA3)
		(qA3) edge node {$1$} (qA4)
		(qA4) edge node {$5$} (qA5)
		(qA5) edge node {$0$} (qA6)
		(qA6) edge [bend right] node {$\eps$} (qS)
		;
	\end{tikzpicture}
	
	\vspace{2em}
	
	\begin{tikzpicture}[auto,>=stealth', node distance=3cm,auto,every state/.style={thick}]
	\node (init) {};
	\node[state] (S) [right=.7cm of init] {$S$};
	\node[state] (q1) [right=3cm of qS] {$q1$};
	\node[state] (q2) [below of=q1] {$q2$};
	\node[state] (q3) [below of=q2] {$q3$};
	\node[state] (q4) [below of=q3, right=0.7cm of q3] {$q4$};
	\node[state,accepting] (q5) [below right of=q4] {$q5$};
	
	\path[->]
	(init) edge (qS)
	(qS) edge [loop above] node {\small{$ 0, 1, 2, 3, 4 $}} (qS)
	
	(qS) edge [bend left] node {$ 5 $} (q1)
	(q1) edge [bend left] node {\small{$ 0, 1, 2, 3, 4 $}} (qS)
	
	(q1) edge [loop above] node {$ 5 $} (q1)
	(q1) edge [bend left] node {$3$} (q2)
	
	(q2) edge [bend left] node {$5$} (q1)
	(q2) edge [bend left=10] node {\small{$0, 2, 3, 4$}} (qS)
	(q2) edge node {$1$} (q3)
	
	(q3) edge node {$5$} (q4)
	(q3) edge [bend left] node[right] {\small{$0, 1, 2, 3, 4$}} (qS)
	
	(q4) edge [bend right] node {$3$} (q2)
	(q4) edge [bend right] node {$5$} (q1)
	(q4) edge [bend left=40] node {\small{$1, 2, 4$}} (qS)
	
	(q4) edge node {$0$} (q5)
	
	(q5) edge [bend right] node {$5$} (q1)
	(q5) edge [bend left=60] node {\small{$0, 1, 2, 3, 4$}} (qS)
	
	
	
	;
	\end{tikzpicture}
	
	
\end{enumerate}

\newpage
\section*{Задание 3}
Условие задания.
\begin{enumerate}[label=(\roman{*})]
	\item \ldots
\end{enumerate}


\end{document}